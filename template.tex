\documentclass[man,12pt]{apa6} % Options: jou (journal), man (manuscript), doc (document)
\usepackage[american]{babel} % Set language
\usepackage{csquotes}        % Required for APA citations
\usepackage{apacite}         % For APA-style references
\usepackage{graphicx}        % For including figures
\usepackage{amsmath}         % For mathematical equations

\title{Your Title Here}
\shorttitle{Short Title} % Optional, used in headers
\author{Your Name}
\affiliation{Your Institution}

\abstract{%
  Write your abstract here. It should summarize the key points of your paper in about 150--250 words.
}

\keywords{keyword1, keyword2, keyword3} % Optional, for indexing

\begin{document}

\maketitle

\section{Introduction}
Introduce your research problem, context, and objectives. Use APA-style citations, for example: \citeA{smith2020} or \cite{johnson2019}.

\section{Method}
Describe your methods, including participants, materials, and procedures.

\subsection{Participants}
Provide information about your sample.

\subsection{Materials}
Describe the materials or datasets used.

\subsection{Procedure}
Explain the steps of your study or analysis.

\section{Results}
Present your findings using text, tables, and figures. Refer to tables and figures in the text (e.g., see Figure~\ref{fig:example}).

\section{Discussion}
Interpret your findings and discuss implications, limitations, and potential future research.

\section{References}
\bibliographystyle{apacite} % APA citation style
\bibliography{references}  % Reference to your .bib file

\appendix
\section{Appendix: Additional Materials}
Include any supplementary information, like questionnaires or raw data summaries.

\end{document}
